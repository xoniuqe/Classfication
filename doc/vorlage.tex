%%%%%%%%%%%%%%%%%%% vorlage.tex %%%%%%%%%%%%%%%%%%%%%%%%%%%%%
%
% LaTeX-Vorlage zur Erstellung von Projekt-Dokumentationen
% im Fachbereich Informatik der Hochschule Trier
%
% Basis: Vorlage svmono des Springer Verlags
%
%%%%%%%%%%%%%%%%%%%%%%%%%%%%%%%%%%%%%%%%%%%%%%%%%%%%%%%%%%%%%

\documentclass[envcountsame,envcountchap, deutsch]{i-studis}

\usepackage{makeidx}         	% Index
\usepackage{multicol}        	% Zweispaltiger Index
%\usepackage[bottom]{footmisc}	% Erzeugung von Fu�noten

%%-----------------------------------------------------
%\newif\ifpdf
%\ifx\pdfoutput\undefined
%\pdffalse
%\else
%\pdfoutput=1
%\pdftrue
%\fi
%%--------------------------------------------------------
%\ifpdf
\usepackage[pdftex]{graphicx}
\usepackage[pdftex,plainpages=false]{hyperref}
%\else
%\usepackage{graphicx}
%\usepackage[plainpages=false]{hyperref}
%\fi

%%-----------------------------------------------------
\usepackage{color}				% Farbverwaltung
%\usepackage{ngerman} 			% Neue deutsche Rechtsschreibung
\usepackage[english, ngerman]{babel}
\usepackage[latin1]{inputenc} 	% Erm�glicht Umlaute-Darstellung
%\usepackage[utf8]{inputenc}  	% Erm�glicht Umlaute-Darstellung unter Linux (je nach verwendetem Format)

%-----------------------------------------------------
\usepackage{listings} 			% Code-Darstellung
\lstset
{
	basicstyle=\scriptsize, 	% print whole listing small
	keywordstyle=\color{blue}\bfseries,
								% underlined bold black keywords
	identifierstyle=, 			% nothing happens
	commentstyle=\color{red}, 	% white comments
	stringstyle=\ttfamily, 		% typewriter type for strings
	showstringspaces=false, 	% no special string spaces
	framexleftmargin=7mm, 
	tabsize=3,
	showtabs=false,
	frame=single, 
	rulesepcolor=\color{blue},
	numbers=left,
	linewidth=146mm,
	xleftmargin=8mm
}
\usepackage{textcomp} 			% Celsius-Darstellung
\usepackage{amssymb,amsfonts,amstext,amsmath}	% Mathematische Symbole
\usepackage[german, ruled, vlined]{algorithm2e}
\usepackage[a4paper]{geometry} % Andere Formatierung
\usepackage{bibgerm}
\usepackage{array}
\hyphenation{Ele-men-tar-ob-jek-te  ab-ge-tas-tet Aus-wer-tung House-holder-Matrix Le-ast-Squa-res-Al-go-ri-th-men} 		% Weitere Silbentrennung bei Bedarf angeben
\setlength{\textheight}{1.1\textheight}
\pagestyle{myheadings} 			% Erzeugt selbstdefinierte Kopfzeile
\makeindex 						% Index-Erstellung


%--------------------------------------------------------------------------
\begin{document}
%------------------------- Titelblatt -------------------------------------
\title{Konzeption und Implementierung eines Systems zur automatischen Analyse und Klassifikation von Berichten in Nachrichtenportalen. }
\subtitle{Conception and implementation of a system for the automatic analysis and classification of articles in newsportals}
%---- Die Art der Dokumentation kann hier ausgew�hlt werden---------------
%\project{Bachelor-Projektarbeit}
\project{Bachelor-Abschlussarbeit}
%\project{Master-Projektstudium}
%\project{Master-Abschlussarbeit}
%\project{Seminar zur Vorlesung ...}
%\project{Hausarbeit zur Vorlesung ...}
%--------------------------------------------------------------------------
\supervisor{Karl Hans Blaesius} 		% Betreuer der Arbeit
\author{Tobias Arens} 							% Autor der Arbeit
\address{Trier,} 							% Im Zusammenhang mit dem Datum wird hinter dem Ort ein Komma angegeben
\submitdate{13.12.2016} 				% Abgabedatum
%\begingroup
%  \renewcommand{\thepage}{title}
%  \mytitlepage
%  \newpage
%\endgroup
\begingroup
  \renewcommand{\thepage}{Titel}
  \mytitlepage
  \newpage
\endgroup
%--------------------------------------------------------------------------
\frontmatter 
%--------------------------------------------------------------------------
\preface

Ein Vorwort ist nicht unbedingt n�tig. Falls Sie ein Vorwort schreiben, so ist dies der Platz, um z.B. die Firma vorzustellen, in der diese Arbeit entstanden ist, oder einigen Leuten zu danken, die in irgendeiner Form positiv zur Entstehung dieser Arbeit beigetragen haben. Auf keinen Fall sollten Sie im Vorwort die Aufgabenstellung n�her erl�utern oder vertieft auf technische Sachverhalte eingehen.				% Vorwort (optional)
\kurzfassung

%% deutsch
\paragraph*{}

In dieser Arbeit wurde eine erweiterte Variante des Klassifikationsalgorithmus Naive-Bayes implementiert. Aus in Textdateien vorliegenden Konfigurationsdaten wird dynamsich ein Klassifikationskorpus erzeugt, der es erm�glicht aktuelle Nachrichtenartikel verschiedener Onlineportale zu klassifizieren.
Mithilfe einer simplen graphischen Benutzeroberfl�che ist es m�glich diese Artikel zu durchsuchen und einen Einblick in die zugrunde liegenden Daten zu gewinnen. 
Ben�tigte Grundlagen der Textklassifikation sowie das Vorgehen bei Konzeption und Realisierung der Arbeit werden ausf�hrlich beschrieben.

%% englisch
\paragraph*{}
This paper shows an implementation of an advanced variant of the Naive-Bayes classification algorithm. Through available configuration files a classification corpus is dynamically generated by which it is possible to classify current news from online news portals. 
A simple graphical-user-interface allows it to filter the results of this search and makes the underlying data easily accessible.
Required fundamentals of text-classification and the approach of design and implementation of this work are described in detail. 			% Kurzfassung Deutsch/English
\tableofcontents 						% Inhaltsverzeichnis
\listoffigures 							% Abbildungsverzeichnis (optional)
\listoftables 							% Tabellenverzeichnis (optional)
%--------------------------------------------------------------------------
\mainmatter                        		% Hauptteil (ab hier arab. Seitenzahlen)
%--------------------------------------------------------------------------
% Die Kapitel werden in separaten .tex-Dateien abgelegt und hier eingebunden.
\chapter{Einleitung und Problemstellung}

%Begonnen werden soll mit einer Einleitung zum Thema, also Hintergrund und Ziel erl�utert werden.

%Weiterhin wird das vorliegende Problem diskutiert: Was ist zu l�sen, warum ist es wichtig, dass man dieses Problem l�st und welche L�sungsans�tze gibt es bereits. Der Bezug auf vorhandene oder eben bisher fehlende L�sungen begr�ndet auch die Intention und Bedeutung dieser Arbeit. Dies k�nnen allgemeine Gesichtspunkte sein: Man liefert einen Beitrag f�r ein generelles Problem oder man hat eine spezielle Systemumgebung oder ein spezielles Produkt (z.B. in einem Unternehmen), woraus sich dieses noch zu l�sende Problem ergibt.

%Im weiteren Verlauf wird die Problemstellung konkret dargestellt: Was ist spezifisch zu l�sen? Welche Randbedingungen sind gegeben und was ist die Zielsetzung? Letztere soll das
%beschreiben, was man mit dieser Arbeit (mindestens) erreichen m�chte.


Zur Informationsbeschaffung in der heutigen Zeit gibt es zahlreiche M�glichkeiten. So gibt es hunderte Online-Nachrichtenportale, Zeitschriften, Zeitungen und Magazine aus denen wertvolle Informationen gewonnen werden kann. 

Diese Diversit�t hat viele positive Eigenschaften, wie das vorliegen verschiedener Standpunkte und die Beleuchtung von Ereignissen und Nachrichten aus vielen Blickrichtungen. Die dadurch erreichte Abdeckung ist begr��enswert schafft allerdings einige Probleme. 
Es ist umst�ndlich bei Interesse an einem bestimmten Thema passende Artikel zu finden da daf�r eine M�glichkigkeit einer gro�fl�chigen Abdeckung verschiedener Portale fehlt.\\


Um Artikel thematisch zuordnen zu k�nnen ist eine automatisierte Klassifizierung sinnvoll, die mithilfe vorab definierter Klassen neue Informationsquellen zielsicher in ebenjene Abbilden kann. Dies erm�glicht eine zielgerichtete, breitgef�cherte Suche nach themenbezogenen Informationen.\\


In dem Rahmen dieser Arbeit wird eine Anwendung entwickelt, die dieses Problem l�sen soll. 
Daf�r sind allerdings mehrere Schritte notwendig: 
Es muss eine Schnittstelle entwickelt werden, die Nachrichtenartikel verschiedener Quellen auslesen kann um diese f�r die weitere Verarbeitung nutzen zu k�nnen.
Die so gewonnenen Daten m�ssen mit entsprechenden Metriken bewertet und mithilfe eines Klassifizierungsalgorithmus korrekt zugeordnet werden k�nnen.
Um dies zu erm�glichen soll ein m�glichst simpler Algorithmus gefunden werden, der eine m�glichst hohe Genauigkeit der Zuordnung erreicht.
Damit die Anwendung bedienbar ist, wird eine graphische Oberfl�che entwickelt, die eine Kategorisierung aktueller Nachrichten erlaubt und einen �berblick �ber die zugrunde liegenden Klassifikationsdaten erm�glicht. \\

Die Arbeit beinhaltet Techniken aus dem Bereich des Text-Mining, welches sich mit M�glichkeiten der Gewinnung von neuen Informationen aus schwach-strukturierten Texten befasst und hat einige �berschneidungen mit der Information Retrieval, das sich damit befasst bestehende Informationen aufzufinden.
\chapter{Weitere Kapitel}

Die Gliederung h�ngt nat�rlich vom Thema und von der L�sungsstrategie ab. Als n�tzliche
Anhaltspunkte k�nnen die Entwicklungsstufen oder - schritte z.B. der Softwareentwicklung betrachtet werden. N�tzliche Gesichtspunkte erh�lt und erkennt man, wenn man sich
\begin{itemize}
  \item in die Rolle des Lesers oder
  \item in die Rolle des Entwicklers, der die Arbeit z.B. fortsetzen, erg�nzen oder pflegen soll,
\end{itemize}
versetzt. In der Regel wird vorausgesetzt, dass die Leser einen fachlichen Hintergrund haben - z.B. Informatik studiert haben. D.h. nur in besonderen, abgesprochenen F�llen schreibt man in popul�rer Sprache, so dass auch Nicht-Fachleute die Ausarbeitung prinzipiell lesen und verstehen k�nnen.

Die �u�ere Gestaltung der Ausarbeitung hinsichtlich Abschnittformate, Abbildungen, mathematische Formeln usw. wird in \hyperref[Stile]{Kapitel~\ref*{Stile}} kurz dargestellt.
\chapter{LaTeX-Bausteine}\label{Stile}

Der Text wird in bis zu drei Ebenen gegliedert:

\begin{enumerate}
  \item Kapitel ( \verb \chapter{Kapitel} ), \index{Kapitel}
  \item Unterkapitel  ( \verb \section{Abschnitt} ) und
  \item Unterunterkapitel  ( \verb \subsection{Unterabschnitte} ).
\end{enumerate}

\section{Abschnitt}\index{Abschnitt}
Text der Gliederungsebene 2.


\subsection{Unterabschnitt} \index{Unterabschnitt}
Text der Gliederungsebene 3.
Text Text Text Text Text Text Text Text Text Text Text Text Text Text Text
Beispiel f�r Quelltext\index{Quelltext} \\[2 ex]
\noindent
\begin{minipage}{1.0\textwidth} \small
\begin{lstlisting}
	Prozess 1:
	
	Acquire();
		a := 1;
	Release();
	...
	Acquire();
	if(b == 0)
	{					
		c := 3;
		d := a;
	}				
	Release();
\end{lstlisting}
\end{minipage}

\vspace{2cm}
\noindent
\begin{minipage}{1.0\textwidth} \small
\begin{lstlisting}
	Prozess 2:
	
	Acquire();
		b := 1;
	Release();
	...
	Acquire();
	if(a == 0)
	{					
		c := 5;
		d := b;
	}				
	Release();
\end{lstlisting}
\end{minipage}
\vskip 1em

Gr��ere Code-Fragmente sollten im Anhang eingef�gt werden.

\section{Abbildungen und Tabellen}

Abbildung\index{Abbildung} und Tabellen\index{Tabelle} werden zentriert eingef�gt. Grunds�tzlich sollen sie
erst dann erscheinen, nach dem sie im Text angesprochen wurden (siehe Abb. \ref{a1}). Abbildungen und Tabellen (siehe Tabelle \ref{t1}) k�nnen
im (flie�enden) Text (\verb here ), am Seitenanfang (\verb top ), am Seitenende
(\verb bottom ) oder auch gesammelt auf einer nachfolgenden Seite (\verb page )
oder auch ganz am Ende der Ausarbeitung erscheinen. Letzteres sollte man nur
dann w�hlen, wenn die Bilder g�nstig zusammen zu betrachten sind und die
Ausarbeitung nicht zu lang ist ($< 20$ Seiten).

%\begin{figure} %[hbtp]
%	\centering
%		\includegraphics{images/p1ReadSeq.pdf}
%	\caption{Bezeichnung der Abbildung}
%	\label{a1}
%\end{figure}

\begin{table} %[hbtp]
	\centering
		\begin{tabular}{l | l l l l}
		\textbf{Prozesse} & \textbf{Zeit} $\rightarrow$ \\
		\hline
			$P_{1}$ & $W(x)1$ \\
			$P_{2}$ & & $W(x)2$ \\
			$P_{3}$ & & $R(x)2$ & & $R(x)1$\\
			$P_{4}$ & & & $R(x)2$ & $R(x)1$\\
		\end{tabular}
	\caption{Bezeichnung der Tabelle}
	\label{t1}
\end{table}


\section{Mathematische Formel}\index{Formel}
Mathematische Formeln bzw. Formulierungen k�nnen sowohl im
laufenden Text (z.B. $y=x^2$) oder abgesetzt und zentriert im Text
erscheinen. Gleichungen sollten f�r Referenzierungen nummeriert
werden (siehe Formel \ref{gl-1}).
\begin{equation}
\label{gl-1}
e_{i}=\sum _{i=1}^{n}w_{i}x_{i}
\end{equation}

Entscheidungsformel:

\begin{equation}
\psi(t)=\left\{\begin{array}{ccc}
1 &  \qquad 0 <= t < \frac{1}{2} \\
-1 &  \qquad \frac{1}{2} <= t <1 \\
0 & \qquad sonst
\end{array} \right.
\end{equation}


Matrix:\index{Matrix}
\begin{equation}
A = \left(
\begin{array}{llll}
a_{11} & a_{12} & \ldots & a_{1n} \\
a_{21} & a_{22} & \ldots & a_{2n} \\
\vdots & \vdots & \ddots & \vdots \\
a_{n1} & a_{n2} & \ldots & a_{nn} \\
\end{array}
\right)
\end{equation}

Vektor:\index{Vektor} 

\begin{equation}
\overline{a} = \left(
\begin{array}{c}
a_{1}\\
a_{2}\\
\vdots\\
a_{n}\\
\end{array}
\right)
\end{equation}

\section{S�tze, Lemmas und Definitionen}\index{Satz}\index{Lemma}\index{Definition}

S�tze, Lemmas, Definitionen, Beweise,\index{Beweis} Beispiele\index{Beispiel} k�nnen in speziell daf�r vorgesehenen Umgebungen erstellt werden.

\begin{definition}(Optimierungsproblem)

Ein \emph{Optimierungsproblem} $\mathcal{P}$ ist festgelegt durch ein Tupel
$(I_\mathcal{P}, sol_\mathcal{P}, m_\mathcal{P}, goal)$ wobei gilt

\begin{enumerate}
\item $I_\mathcal{P}$ ist die Menge der Instanzen,
\item $sol_\mathcal{P} : I_\mathcal{P} \longmapsto \mathbb{P}(S_\mathcal{P})$ ist eine Funktion, die jeder Instanz $x \in I_\mathcal{P}$ eine Menge zul�ssiger L�sungen zuweist,
\item $m_\mathcal{P} : I_\mathcal{P} \times S_\mathcal{P} \longmapsto \mathbb{N}$ ist eine Funktion, die jedem Paar $(x,y(x))$ mit $x \in I_\mathcal{P}$ und $y(x) \in sol_\mathcal{P}(x)$ eine
Zahl $m_\mathcal{P}(x,y(x)) \in \mathbb{N}$ zuordnet (= Ma� f�r die L�sung $y(x)$ der Instanz $x$), und
\item $goal \in \{min,max\}$.
\end{enumerate}

\end{definition}

\begin{example} MINIMUM TRAVELING SALESMAN (MIN-TSP)
\begin{itemize}
\item $I_{MIN-TSP} =_{def}$ s.o., ebenso $S_{MIN-TSP}$
\item $sol_{MIN-TSP}(m,D) =_{def} S_{MIN-TSP} \cap \mathbb{N}^m$ 
\item $m_{MIN-TSP}((m,D),(c_1, \ldots , c_m)) =_{def} \sum_{i=1}^{m-1} D(c_i, c_{i+1}) + D(c_m,c_1)$ 
\item $goal_{MIN-TSP} =_{def} min$
\end{itemize}
\begin{flushright}
$\qed$
\end{flushright}
\end{example}

\begin{theorem} Sei $\mathcal{P}$ ein \textbf{NP}-hartes Optimierungsproblem.
Wenn $\mathcal{P} \in$ \textbf{PO}, dann ist \textbf{P} = \textbf{NP}.
\end{theorem}

\begin{proof} Um zu zeigen, dass \textbf{P} = \textbf{NP} gilt, gen�gt es
wegen Satz A.30 zu zeigen, dass ein einziges \textbf{NP}-vollst�ndiges
Problem in \textbf{P} liegt. Sei also $\mathcal{P}'$ ein beliebiges \textbf{NP}-vollst�ndiges Problem.

Weil $\mathcal{P}$ nach Voraussetzung \textbf{NP}-hart ist, gilt insbesondere
$\mathcal{P}' \leq_T \mathcal{P}_C$. Sei $R$ der zugeh�rige
Polynomialzeit-Algorithmus dieser Turing-Reduktion.
Weiter ist $\mathcal{P} \in$ \textbf{PO} vorausgesetzt, etwa verm�ge eines
Polynomialzeit-Algorithmus $A$. Aus den beiden
Polynomialzeit-Algorithmen $R$ und $A$ erh�lt man nun
leicht einen effizienten Algorithmus f�r $\mathcal{P}'$: Ersetzt man
in $R$ das Orakel durch $A$, ergibt dies insgesamt eine polynomielle
Laufzeit. 
%\begin{flushright}
$\qed$
% \end{flushright}
\end{proof}

\begin{lemma} Aus \textbf{PO} $=$ \textbf{NPO} folgt \textbf{P} $=$ \textbf{NP}.
\end{lemma}

\begin{proof} Es gen�gt zu zeigen, dass unter der angegeben
Voraussetzung KNAPSACK $\in$ \textbf{P} ist.

Nach Voraussetung ist MAXIMUM KNAPSACK $\in$ \textbf{PO},
d.h. die Berechnung von $m^*(x)$ f�r jede Instanz $x$ ist
in Polynomialzeit m�glich. Um KNAPSACK bei Eingabe
$(x,k)$ zu entscheiden, m�ssen wir nur noch $m^*(x) \geq k$
pr�fen. Ist das der Fall, geben wir $1$, sonst $0$ aus. Dies
bleibt insgesamt ein Polynomialzeit-Algorithmus. 
\begin{flushright}
$\qed$
\end{flushright}
\end{proof}

\section{Fu�noten}

In einer Fu�note k�nnen erg�nzende Informationen\footnote{Informationen die f�r die Arbeit zweitrangig sind, jedoch f�r den Leser interessant sein k�nnten.} angegeben werden. Au�erdem kann eine Fu�note auch Links enthalten. Wird in der Arbeit eine Software (zum Beispiel Java-API\footnote{\url{http://java.sun.com/}}) eingesetzt, so kann die Quelle, die diese Software zur Verf�gung stellt in der Fu�note angegeben werden.

\section{Literaturverweise}\index{Literatur}
Alle benutzte Literatur wird im Literaturverzeichnis angegeben\footnote{Dazu wird ein sogennanter bib-File, literatur.bib verwendet.}. Alle angegebene Literatur sollte mindestens einmal im Text referenziert werden\cite{Coulouris:02}.
\chapter{Beispiel-Kapitel}

In diesem Kapitel wird beschrieben, warum es unterschiedliche Konsistenzmodelle\index{Konsistenzmodelle} gibt. Au�erdem werden die Unterschiede zwischen strengen Konsistenzmodellen\index{Linearisierbarkeit} (Linearisierbarkeit, sequentielle Konsistenz)\index{sequentiell!Konsistenz} und schwachen Konsistenzmodellen\index{Konsistenz!schwach} (schwache Konsistenz, Freigabekonsistenz)\index{Freigabekonsistenz} erl�utert. Es wird gekl�rt, was Strenge und Kosten (billig, teuer) in Zusammenhang mit Konsistenzmodellen bedeuten.

\section{Warum existieren unterschiedliche Konsistenzmodelle?}

Laut \cite{Malte:97} sind mit der\index{Replikation} Replikation von Daten immer zwei gegens�tzliche Ziele verbunden: die Erh�hung der\index{Verf�gbarkeit} Verf�gbarkeit und die Sicherung der\index{Konsistenz} Konsistenz der Daten. Die Form der Konsistenzsicherung bestimmt dabei, inwiefern das eine Kriterium erf�llt und das andere dementsprechend nicht erf�llt ist (Trade-off zwischen Verf�gbarkeit und der Konsistenz der Daten). Stark konsistente Daten sind stabil, das hei�t, falls mehrere Kopien der Daten existieren, d�rfen keine Abweichungen auftreten. Die Verf�gbarkeit der Daten ist hier jedoch stark eingeschr�nkt. Je schw�cher die Konsistenz wird, desto mehr Abweichungen k�nnen zwischen verschiedenen Kopien einer Datei auftreten, wobei die Konsistenz nur an bestimmten Synchronisationspunkten gew�hrleistet wird. Daf�r steigt aber die Verf�gbarkeit der Daten, weil sie sich leichter replizieren lassen.

Nach \cite{Mosberger:93} kann die Performanzsteigerung der schw�cheren Konsistenzmodelle wegen der Optimierung\index{Optimierung} (Pufferung, Code-Scheduling, Pipelines) 10-40 Prozent betragen. Wenn man bedenkt, dass mit der Nutzung der vorhandenen Synchronisierungsmechanismen schw�chere Konsistenzmodelle den Anforderungen der strengen Konsistenz gen�gen, stellt sich der h�here programmiertechnischer Aufwand bei der Implementierung der schw�cheren Konsistenzmodelle als ihr einziges Manko dar.

In \cite{Cheriton:85} ist beschrieben, wie man sich Formen von DSM vorstellen k�nnte, f�r die ein beachtliches Ma� an\index{Inkonsistenz} Inkonsistenz akzeptabel w�re. Beispielsweise k�nnte DSM verwendet werden, um die Auslastung von Computern in einem Netzwerk zu speichern, so dass Clients f�r die Ausf�hrung ihrer Applikationen die am wenigsten ausgelasteten Computer ausw�hlen k�nnen. Weil die Informationen dieser Art innerhalb k�rzester Zeit ungenau werden k�nnen (und durch die Verwendung der veralteten Daten keine gro�en Nachteile entstehen k�nnen), w�re es vergebliche M�he, sie st�ndig f�r alle Computer im System konsistent zu halten \cite{Coulouris:02}. Die meisten Applikationen stellen jedoch strengere Konsistenzanforderungen.

\section{Klassifizierung eines Konsistenzmodells}

Die zentrale Frage, die f�r die Klassifizierung\index{streng}\index{schwach} (streng oder schwach) eines Konsistenzmodells von Bedeutung ist \cite{Coulouris:02}: wenn ein Lesezugriff auf eine Speicherposition erfolgt, welche Werte von Schreibzugriffen auf diese Position sollen dann dem Lesevorgang bereitgestellt werden? Die Antwort f�r das schw�chste Konsistenzmodell lautet: von jedem Schreibvorgang, der vor dem Lesen erfolgt ist, oder in der "`nahen"' Zukunft, innerhalb des definierten Betrachtungsraums, erfolgten wird. Also irgendein Wert, der vor oder nach dem Lesen geschrieben wurde.

F�r das strengste Konsistenzmodell, Linearisierbarkeit (atomic consistency), stehen alle geschriebenen Werte allen Prozessoren sofort zur Verf�gung: eine Lese-Operation gibt den aktuellsten Wert zur�ck, der geschrieben wurde, bevor das Lesen stattfand. Diese Definition ist aber in zweierlei Hinsicht problematisch. Erstens treten weder Schreib- noch Lese-Operationen zu genau einem Zeitpunkt auf, deshalb ist die Bedeutung von "`aktuellsten"' nicht immer klar. Zweitens ist es nicht immer m�glich, genau festzustellen, ob ein Ereignis vor einem anderen stattgefunden hat, da es Begrenzungen daf�r gibt, wie genau Uhren in einem verteilten System synchronisiert werden k�nnen.

Nachfolgend werden einige Konsistenzmodelle absteigend nach ihrer Strenge vorgestellt. Zuvor m�ssen wir allerdings kl�ren, wie die Lese- und Schreibe-Operationen in dieser Ausarbeitung dargestellt werden.

Sei $x$ eine Speicherposition, dann k�nnen Instanzen dieser Operationen wie folgt ausgedr�ckt werden:
\begin{itemize}
	\item $R(x)a$ - eine Lese-Operation\index{Operation!Lese}, die den Wert $a$ von der Position $x$ liest.
	\item $W(x)b$ - eine Schreib-Operation\index{Operation!Schreib}, die den Wert $b$ an der Position $x$ speichert.
\end{itemize}

\section{Linearisierbarkeit\index{Linearisierbarkeit} (atomic consistency)}

Die Linearisierbarkeit im Zusammenhang mit DSM kann wie folgt definiert werden:
\begin{itemize}
	\item Die verzahnte Operationsabfolge findet so statt: wenn $R(x)a$ in der Folge vorkommt, dann ist die letzte Schreib-Operation, die vor ihr in der verzahnten Abfolge auftritt, $W(x)a$, oder es tritt keine Schreib-Operation vor ihr auf und $a$ ist der Anfangswert von $x$. Das bedeutet, dass eine Variable nur durch eine Schreib-Operation ge�ndert werden kann.
	\item Die Reihenfolge der Operationen in der Verzahnung ist konsistent zu den \underline{Echtzeiten}\index{Echtzeiten}, zu denen die Operationen bei der tats�chlichen Ausf�hrung aufgetreten sind.
\end{itemize}

Die Bedeutung dieser Definition kann an folgendem Beispiel (Tabelle \ref{tab:1}) nachvollzogen werden. Es sei angenommen, dass alle Werte mit $0$ vorinitialisiert sind.

\begin{table}
	\centering
		\begin{tabular}{l | l l l l}
			\textbf{Prozesse} & \textbf{Zeit} $\rightarrow$ & \\
			\hline
			$P_{1}$ & $W(x)1$ & & $W(y)2$ \\
			$P_{2}$ & & $R(x)1$ & & $R(y)2$ \\
		\end{tabular}
	\caption{Linearisierbarkeit ist erf�llt}
	\label{tab:1}
\end{table}

Hier sind beide Bedingungen erf�llt, da die Lese-Operationen den zuletzt geschriebenen Wert zur�ckliefern. Interessanter ist es, zu sehen, wann die Linearisierbarkeit verletzt ist.

\begin{table}
	\centering
		\begin{tabular}{l | l l l l}
		\textbf{Prozesse} & \textbf{Zeit} $\rightarrow$ \\
		\hline
		$P_{1}$ & $W(x)1$ & $W(x)2$ \\
		$P_{2}$ & & & \color{red} $R(x)0$ & \color{black} $R(x)2$ \\
		\end{tabular}
	\caption{Linearisierbarkeit ist verletzt, sequentielle Konsistenz ist erf�llt.}
	\label{tab:2}
\end{table}

In diesem Beispiel (Tabelle \ref{tab:2}) ist die Echtzeit-Anforderung verletzt, da der Prozess $P_{2}$ immer noch den alten Wert liest, obwohl er von Prozess $P_{1}$ bereits ge�ndert wurde. Diese Ausf�hrung w�re aber sequentiell konsistent (siehe kommender Abschnitt), da es eine Verzahnung der Operationen gibt, die diese Werte liefern k�nnte ($R(x)0$, $W(x)1$, $W(x)2$, $R(y)2$). W�rde man beide Lese-Operationen des 2. Prozesses vertauschen, wie in der Tabelle \ref{tab:3} dargestellt, so w�re keine sinnvolle Verzahnung mehr m�glich.

\begin{table}
	\centering
		\begin{tabular}{l | l l l l}
		\textbf{Prozesse} & \textbf{Zeit} $\rightarrow$ \\
		\hline
		$P_{1}$ & $W(x)1$ & $W(x)2$ \\
		$P_{2}$ & & & \color{red} $R(x)2$ &  \color{red} $R(x)0$ \\
			
		\end{tabular}
	\caption{Linearisierbarkeit und sequentielle Konsistenz sind verletzt.}
	\label{tab:3}
\end{table}

In diesem Beispiel sind beide Bedingungen verletzt. Selbst wenn die Echtzeit, zu der die Operationen stattgefunden haben, ignoriert wird, gibt es keine Verzahnung einzelner Operationen, die der Definition entsprechen w�rde.
\chapter{Zusammenfassung}

%In diesem Kapitel soll die Arbeit noch einmal kurz zusammengefasst werden. Insbesondere sollen die wesentlichen Ergebnisse Ihrer Arbeit herausgehoben werden. Erfahrungen, die z.B. Benutzer mit der Mensch-Maschine-Schnittstelle gemacht haben oder Ergebnisse von Leistungsmessungen sollen an dieser Stelle pr�sentiert werden. Sie k�nnen in diesem Kapitel auch die Ergebnisse oder das Arbeitsumfeld Ihrer Arbeit kritisch bewerten. W�nschenswerte Erweiterungen sollen als Hinweise auf weiterf�hrende Arbeiten erw�hnt werden.

Das in dieser Arbeit erstellte Programm erm�glicht es, erfolgreich Artikel in passende Kategorien einzuordnen und somit eine gute Filterung von Nachrichtenartikel zu erzielen.
Problematisch ist diese Kategorisierung allerdings bez�glich der Auswahl der Trainingsdaten. So ist zum einen eine sehr gro�e Menge an Artikeln notwendig um eine genaue Klassifikation zu erm�glichen, doch ist dies auch mit einem gro�en Aufwand verbunden. Die in dieser Arbeit erstellten Trainingsdaten sind nicht Umfangreich genug um eine wirklich zufriedenstellende Kategorisierung zu erm�glichen, zeigen aber auch das Potenzial des Algorithmus.\\

Insgesamt l�sst sich diese Arbeit als Erfolg werten. Die zugrunde liegenden Algorithmen sind voll funktionsf�hig und die Aufteilung der Programmpakete erm�glicht eine einfache Wartung und Erweiterung.
Der intensive Kontakt mit verschiedenen Verfahren zur Textklassifikation und anderen Data-Mining Algorithmen hat bei mir die Absicht geweckt sich noch weiter mit diesem Themengebiet zu besch�ftigen und die genannten Verbesserungen des Programms in Zukunft zu realisieren.

%ausf�hrlicher?

\newpage
\chapter{Ausblick}

Die erstellte Anwendung hat viel Potenzial f�r Verbesserungen, so wird beispielsweise der erstellte Klassifikationskorpus nicht gespeichert und muss bei jedem Programmstart neu erstellt werden, was mit einer mehrmin�tigen Wartezeit verbunden ist.
Desweiteren ist die Abdeckung von nur zwei Nachrichtenportalen nicht optimal, kann aber durch die hier entworfene Bezeichnungssprache relativ einfach erweitert werden.\\
Eine weitere m�gliche Verbesserung w�re es, die schon durchsuchten Artikel abzuspeichern und somit eine M�glichkeit zur Archivierung und langfristigen Verbesserung der Trainingsdaten zu erm�glichen. Die Suche k�nnte ebenso ausgeweitet werden, da sie sich zurzeit nur auf die auf der ersten Seite einer angegebenen Nachrichtenseite bezieht und �ltere Artikel nicht durchsucht. Dies greift mit der erw�hnten Archivierung zusammen, da so schon klassifizierte Artikel nicht erneut gepr�ft werden m�ssten.\\

Denkbar w�re auch eine Erweiterung die es erm�glicht auch andere Quellen zu klassifizieren, beispielsweise eingescannte Dokumente oder Magazine die mithilfe einer Texterkennungssoftware vorverarbeitet werden. So w�rde das Programm nicht nur von Online-Medien abh�ngig sein und w�rde Optionen bieten ein umfassendes Archiv aufzubauen.
Hierf�r k�nnte eine erweiterte Hierarchie sinnvoll sein, die es erlaubt auch Ober- beziehungsweise Unterkategorien zu definieren, mit denen die Kategorisierung und Suche verfeinert werden k�nnte.\\

Da der Klassifikationsalgorithmus mit einem simplen \textit{Bag-of-Words}-Modell arbeitet, der lediglich einzelne W�rter verwendet, k�nnte auch �ber eine m�gliche Verwendung von \textit{N-Grammen} nachgedacht werden. Dies bedeutet statt jeweils ein Wort zu z�hlen, \textit{N} W�rter zu z�hlen und dann zu klassifizieren. Es k�nnte ausgewertet werden ob dies eine Verbesserung erbringt oder nicht. 
Insgesamt k�nnen basierend auf dieser Arbeit viele solcher Optionen ber�cksichtigt werden. Die verwendeten Metriken k�nnen mit der im Kapitel \ref{chap-ext-bayes} beschriebenen Optionen ausgetauscht werden und dann mithilfe von weiteren Testdaten die f�r Nachrichtenartikel beste Methode zu evaluieren.


% ...
%--------------------------------------------------------------------------
\backmatter                        		% Anhang
%-------------------------------------------------------------------------
\bibliographystyle{geralpha}			% Literaturverzeichnis
\bibliography{literatur}     			% BibTeX-File literatur.bib
%--------------------------------------------------------------------------
\printindex 							% Index (optional)
%--------------------------------------------------------------------------
\begin{appendix}						% Anh�nge sind i.d.R. optional
   \chapter{Glossar}

\abbreviation{DisASTer}		{DisASTer (Distributed Algorithms Simulation Terrain), A platform for the Implementation of Distributed Algorithms}
\abbreviation{DSM}			{Distributed Shared Memory}
\abbreviation{AC}			{Linearisierbarkeit (atomic consistency)}
\abbreviation{SC}			{Sequentielle Konsistenz (sequential consistency)}
\abbreviation{WC}			{Schwache Konsistenz (weak consistency)}
\abbreviation{RC}			{Freigabekonsistenz (release consistency)}
			% Glossar   
   \chapter{Erkl�rung der Kandidatin / des Kandidaten}

\begin{description}[$\Box$~]
\item[$\Box$] Die Arbeit habe ich selbstst�ndig verfasst und keine anderen als die angegebenen Quellen- und Hilfsmittel verwendet.\\

\item[$\Box$] Die Arbeit wurde als Gruppenarbeit angefertigt. Meine eigene Leistung ist\\
...\\

Diesen Teil habe ich selbstst�ndig verfasst und keine anderen als die angegebenen Quellen und Hilfsmittel verwendet. \\

Namen der Mitverfasser: ...

\end{description}

\vspace{2cm}

\begin{minipage}[t]{3cm}
\rule{3cm}{0.5pt}
Datum
\end{minipage}
\hfill
\begin{minipage}[t]{9cm}
\rule{9cm}{0.5pt}
Unterschrift der Kandidatin / des Kandidaten
\end{minipage}	% Selbstst�ndigkeitserkl�rung
\end{appendix}

\end{document}
