\chapter{Zusammenfassung}

%In diesem Kapitel soll die Arbeit noch einmal kurz zusammengefasst werden. Insbesondere sollen die wesentlichen Ergebnisse Ihrer Arbeit herausgehoben werden. Erfahrungen, die z.B. Benutzer mit der Mensch-Maschine-Schnittstelle gemacht haben oder Ergebnisse von Leistungsmessungen sollen an dieser Stelle pr�sentiert werden. Sie k�nnen in diesem Kapitel auch die Ergebnisse oder das Arbeitsumfeld Ihrer Arbeit kritisch bewerten. W�nschenswerte Erweiterungen sollen als Hinweise auf weiterf�hrende Arbeiten erw�hnt werden.

Das in dieser Arbeit erstellte Programm erm�glicht es, erfolgreich Artikel in passende Kategorien einzuordnen und somit eine gute Filterung von Nachrichtenartikel zu erzielen.
Problematisch ist diese Kategorisierung allerdings bez�glich der Auswahl der Trainingsdaten. So ist zum einen eine sehr gro�e Menge an Artikeln notwendig um eine genaue Klassifikation zu erm�glichen, doch ist dies auch mit einem gro�en Aufwand verbunden. Die in dieser Arbeit erstellten Trainingsdaten sind nicht Umfangreich genug um eine wirklich zufriedenstellende Kategorisierung zu erm�glichen, zeigen aber auch das Potenzial des Algorithmus.\\

Insgesamt l�sst sich diese Arbeit als Erfolg werten. Die zugrunde liegenden Algorithmen sind voll funktionsf�hig und die Aufteilung der Programmpakete erm�glicht eine einfache Wartung und Erweiterung.
Der intensive Kontakt mit verschiedenen Verfahren zur Textklassifikation und anderen Data-Mining Algorithmen hat bei mir die Absicht geweckt sich noch weiter mit diesem Themengebiet zu besch�ftigen und die genannten Verbesserungen des Programms in Zukunft zu realisieren.

%ausf�hrlicher?

\newpage
\chapter{Ausblick}

Die erstellte Anwendung hat viel Potenzial f�r Verbesserungen, so wird beispielsweise der erstellte Klassifikationskorpus nicht gespeichert und muss bei jedem Programmstart neu erstellt werden, was mit einer mehrmin�tigen Wartezeit verbunden ist.
Desweiteren ist die Abdeckung von nur zwei Nachrichtenportalen nicht optimal, kann aber durch die hier entworfene Bezeichnungssprache relativ einfach erweitert werden.\\
Eine weitere m�gliche Verbesserung w�re es, die schon durchsuchten Artikel abzuspeichern und somit eine M�glichkeit zur Archivierung und langfristigen Verbesserung der Trainingsdaten zu erm�glichen. Die Suche k�nnte ebenso ausgeweitet werden, da sie sich zurzeit nur auf die auf der ersten Seite einer angegebenen Nachrichtenseite bezieht und �ltere Artikel nicht durchsucht. Dies greift mit der erw�hnten Archivierung zusammen, da so schon klassifizierte Artikel nicht erneut gepr�ft werden m�ssten.\\

Denkbar w�re auch eine Erweiterung die es erm�glicht auch andere Quellen zu klassifizieren, beispielsweise eingescannte Dokumente oder Magazine die mithilfe einer Texterkennungssoftware vorverarbeitet werden. So w�rde das Programm nicht nur von Online-Medien abh�ngig sein und w�rde Optionen bieten ein umfassendes Archiv aufzubauen.
Hierf�r k�nnte eine erweiterte Hierarchie sinnvoll sein, die es erlaubt auch Ober- beziehungsweise Unterkategorien zu definieren, mit denen die Kategorisierung und Suche verfeinert werden k�nnte.\\

Da der Klassifikationsalgorithmus mit einem simplen \textit{Bag-of-Words}-Modell arbeitet, der lediglich einzelne W�rter verwendet, k�nnte auch �ber eine m�gliche Verwendung von \textit{N-Grammen} nachgedacht werden. Dies bedeutet statt jeweils ein Wort zu z�hlen, \textit{N} W�rter zu z�hlen und dann zu klassifizieren. Es k�nnte ausgewertet werden ob dies eine Verbesserung erbringt oder nicht. 
Insgesamt k�nnen basierend auf dieser Arbeit viele solcher Optionen ber�cksichtigt werden. Die verwendeten Metriken k�nnen mit der im Kapitel \ref{chap-ext-bayes} beschriebenen Optionen ausgetauscht werden und dann mithilfe von weiteren Testdaten die f�r Nachrichtenartikel beste Methode zu evaluieren.

