\chapter{Einleitung und Problemstellung}

%Begonnen werden soll mit einer Einleitung zum Thema, also Hintergrund und Ziel erl�utert werden.

%Weiterhin wird das vorliegende Problem diskutiert: Was ist zu l�sen, warum ist es wichtig, dass man dieses Problem l�st und welche L�sungsans�tze gibt es bereits. Der Bezug auf vorhandene oder eben bisher fehlende L�sungen begr�ndet auch die Intention und Bedeutung dieser Arbeit. Dies k�nnen allgemeine Gesichtspunkte sein: Man liefert einen Beitrag f�r ein generelles Problem oder man hat eine spezielle Systemumgebung oder ein spezielles Produkt (z.B. in einem Unternehmen), woraus sich dieses noch zu l�sende Problem ergibt.

%Im weiteren Verlauf wird die Problemstellung konkret dargestellt: Was ist spezifisch zu l�sen? Welche Randbedingungen sind gegeben und was ist die Zielsetzung? Letztere soll das
%beschreiben, was man mit dieser Arbeit (mindestens) erreichen m�chte.


Zur Informationsbeschaffung in der heutigen Zeit gibt es zahlreiche M�glichkeiten. So gibt es hunderte Online-Nachrichtenportale, Zeitschriften, Zeitungen und Magazine aus denen wertvolle Informationen gewonnen werden kann. 

Diese Diversit�t hat viele positive Eigenschaften, wie das vorliegen verschiedener Standpunkte und die Beleuchtung von Ereignissen und Nachrichten aus vielen Blickrichtungen. Die dadurch erreichte Abdeckung ist begr��enswert schafft allerdings einige Probleme. 
Es ist umst�ndlich bei Interesse an einem bestimmten Thema passende Artikel zu finden da daf�r eine M�glichkigkeit einer gro�fl�chigen Abdeckung verschiedener Portale fehlt.\\


Um Artikel thematisch zuordnen zu k�nnen ist eine automatisierte Klassifizierung sinnvoll, die mithilfe vorab definierter Klassen neue Informationsquellen zielsicher in ebenjene Abbilden kann. Dies erm�glicht eine zielgerichtete, breitgef�cherte Suche nach themenbezogenen Informationen.\\


In dem Rahmen dieser Arbeit wird eine Anwendung entwickelt, die dieses Problem l�sen soll. 
Daf�r sind allerdings mehrere Schritte notwendig: 
Es muss eine Schnittstelle entwickelt werden, die Nachrichtenartikel verschiedener Quellen auslesen kann um diese f�r die weitere Verarbeitung nutzen zu k�nnen.
Die so gewonnenen Daten m�ssen mit entsprechenden Metriken bewertet und mithilfe eines Klassifizierungsalgorithmus korrekt zugeordnet werden k�nnen.
Um dies zu erm�glichen soll ein m�glichst simpler Algorithmus gefunden werden, der eine m�glichst hohe Genauigkeit der Zuordnung erreicht.
Damit die Anwendung bedienbar ist, wird eine graphische Oberfl�che entwickelt, die eine Kategorisierung aktueller Nachrichten erlaubt und einen �berblick �ber die zugrunde liegenden Klassifikationsdaten erm�glicht. \\

Die Arbeit beinhaltet Techniken aus dem Bereich des Text-Mining, welches sich mit M�glichkeiten der Gewinnung von neuen Informationen aus schwach-strukturierten Texten befasst und hat einige �berschneidungen mit der Information Retrieval, das sich damit befasst bestehende Informationen aufzufinden.